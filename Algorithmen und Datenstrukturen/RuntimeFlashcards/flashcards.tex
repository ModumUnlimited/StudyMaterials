\documentclass[avery5371,grid]{flashcards}

\usepackage[dvipsnames,usenames, table]{xcolor} 
\usepackage[utf8]{inputenc}
\usepackage[T1]{fontenc}\usepackage{amsfonts}
\usepackage{amsmath}
\cardfrontstyle[\large\slshape]{headings}
\cardbackstyle{empty}
%BEGIN LISTINGDEF
\usepackage{listings}

\usepackage{hyperref}


\renewcommand{\ttdefault}{pcr}

\definecolor{listing-background}{rgb}{0.97,0.97,0.97}
\definecolor{listing-rule}{HTML}{B3B2B3}
\definecolor{listing-numbers}{HTML}{B3B2B3}
\definecolor{listing-text-color}{HTML}{000000}
\definecolor{listing-keyword}{HTML}{435489}
\definecolor{listing-identifier}{HTML}{435489}
\definecolor{listing-string}{HTML}{00999a}
\definecolor{listing-comment}{HTML}{8e8e8e}
\definecolor{listing-javadoc-comment}{HTML}{006CA9}

\lstdefinestyle{eisvogellistingstyle}{
	language=java,
	numbers=none,
	backgroundcolor=\color{listing-background},
	basicstyle=\color{listing-text-color}\footnotesize\ttfamily{}, % print whole listing small
	xleftmargin=0.8em, % 2.8 with line numbers
	breaklines=true,
	frame=single,
	framesep=0.6mm,
	rulecolor=\color{listing-rule},
	frameround=ffff,
	framexleftmargin=0.4em, % 2.4 with line numbers | 0.4 without them
	tabsize=4, %width of tabs
	numberstyle=\color{listing-numbers},
	aboveskip=1.0em,
	keywordstyle=\color{listing-keyword}\bfseries, % underlined bold black keywords
	classoffset=0,
	sensitive=true,
	identifierstyle=\color{listing-identifier}, % nothing happens
	commentstyle=\color{listing-comment}, % white comments
	morecomment=[s][\color{listing-javadoc-comment}]{/**}{*/},
	stringstyle=\color{listing-string}, % typewriter type for strings
	showstringspaces=false, % no special string spaces
	escapeinside={/*@}{@*/}, % for comments
}
\lstset{style=eisvogellistingstyle}
%END LISTINGDEF

\begin{document}
\footnotesize


%    to use code use the template below!
%\begin{lstlisting}[language=Java]
%Random randomGenerator = new Random();
%int rand = randomGenerator.nextInt(bound);
%\end{lstlisting}
%



\begin{flashcard}[A\&D]{Adjazenzliste vs. Matrix \\{\small \begin{itemize}
\item Speicher
\item Alle Nachbarn von $v \in V$
\item Ist $(u,v)\in E$?
\item Kante einfügen
\item Kante löschen    
\end{itemize}}}
\vspace*{\stretch{1}}
    \begin{tabular}{| c | c | c |}
        \textsc{Operation} & \textsc{Adj. Matrix} & \textsc{Adj. Liste}  \\ \hline
        Speicher &$\mathcal{O}(|V|^2)$ & $\mathcal{O}(|V| + |E|)$ \\ 
        Alle Nachbarn von $v \in V$ & $\mathcal{O}(|V|)$ & $\mathcal{O}(deg^{+}(v))$ \\
        Ist $(u,v)\in E$? & $\mathcal{O}(1)$ &$\mathcal{O}(deg^{+}(v))$ \\
        Kante einfügen &  $\mathcal{O}(1)$ & $\mathcal{O}(deg^{+}(v))$\\
        Kante löschen & $\mathcal{O}(1)$  &  $\mathcal{O}(deg^{+}(v))$
    \end{tabular}
\vspace*{\stretch{1}}
\end{flashcard}






\begin{flashcard}[A\&D]{Tiefensuche}
    \vspace*{\stretch{1}}
        Wir besuchen bei der Tiefensuche jeden Knoten. Somit sicherlich ${\Theta}(|V|)$. Für jeden Knoten fügen wir ${\Theta}(deg^{+}(v))$ Kanten hinzu für welche wir prüfen ob schon besucht wurden in konstanter Zeit.
        \par 
        Somit ${\Theta}(|V| + \sum_{v\in V} (deg^{+}(v) + c) ) = {\Theta}(|V| + |E|)$.
    \vspace*{\stretch{1}}
\end{flashcard}

\begin{flashcard}[A\&D]{Breitensuche}
    \vspace*{\stretch{1}}
        Wir besuchen bei der Breitensuche jeden Knoten. Somit sicherlich ${\Theta}(|V|)$. Für jeden Knoten fügen wir ${\Theta}(deg^{+}(v))$ Kanten hinzu für welche wir prüfen ob schon besucht wurden in konstanter Zeit.
        \par 
        Somit ${\Theta}(|V| + \sum_{v\in V} (deg^{+}(v) + c) ) = {\Theta}(|V| + |E|)$.
    \vspace*{\stretch{1}}
\end{flashcard}



\begin{flashcard}[A\&D]{Dijkstra \\  {\small mit MinHeap (da Fib. Heap sehr kompliziert)}}
    \vspace*{\stretch{1}}
    Initialisiere alle Distanzen in leerem Array in $\Theta(|V|)$.
    Nun gehen in \textsc{while} Schleife die solange läuft bis alle Knoten abgearbeitet sind $\Theta(|V|)$. In jedem Schleifendurchgang werden vom zum entfernenden Knoten alle ausgehenden Kanten angeschaut und im schlimmsten Fall werden entweder \textsc{DecreaseKey} oder \textsc{Insert} für den Endknoten der Kante aufgerufen.
    Da wir ja einen Knoten entfernt haben müssen wir mit \textsc{Rebalance} den Heap wiederherstellen. Somit rufen wir total $|E|$ mal \textsc{Rebalance} auf und $|V|$ mal \textsc{DecreaseKey} oder \textsc{Insert}. In einem \href{https://github.com/ModumUnlimited/DavesAlgorithmCollection/blob/master/datastructures/DecreaseKeyMinHeap.java}{modifiziertem MinHeap} haben alle diese Funktionen eine Laufzeit von $\mathcal{O}(log(|V|))$


    Somit ${\Theta}(|V| + |V|\log|V| + |E|\log|V|) = {\Theta}((|V|+ |E|)\log |V|)$ mit Minheap und ${\Theta(|V|\log(|V|)+|E|)}$ mit Fib. Heap.
    \vspace*{\stretch{1}}
\end{flashcard}






\begin{flashcard}[A\&D]{Bellman-Ford}
    \vspace*{\stretch{1}}
    Wir ein Array der Grösse $|V|$ in insgesamt $|V|-1$ Durchgängen schauen wir für alle $e\in E$ ob wir den Distanzwert des Ziels verringern können. Somit ist die Laufzeit $\Theta(|V||E|)$
    \vspace{0.5cm}
    Alternativ nach Skript: Wir haben eine $|V|^2$ Tabelle und jeder Eintrag hängt von $|V|$ Vorgänger ab. Dann ist $\Theta(|V^3|)$.
    \vspace*{\stretch{1}}
\end{flashcard}

\begin{flashcard}[A\&D]{Floyd-Warshall}
    \vspace*{\stretch{1}}
Initialisiere $|V|^2$ Tabelle, trage $\infty$ ein wo es keine Kante gibt und Kantengewicht wo es Kanten gibt. Nun "relaxiere" $|V|$ mal. Das gibt total $\Theta(V^3)$. 
\par Super video: \href{https://youtu.be/I1kA7c09lZE}{https://youtu.be/I1kA7c09lZE}.
    \vspace*{\stretch{1}}
\end{flashcard}

\begin{flashcard}[A\&D]{Johnson's Algorithm}
    \vspace*{\stretch{1}}
    \begin{itemize}
        \item Füge Knoten $v_{new}$ hinzu und verbinde mit allen anderen mit Kantengewicht 0. $\mathcal{O}(|V|)$
        \item Starte Bellman-Ford von $v_{new}$ $\mathcal{O}(|V||E|)$
        \item Transformiere Kantengewichte mit ${c'}_e = c_e + p_u - p_v $ für alle $(u,v)\in E$,  $\mathcal{O}(|V|)$
        \item Führe nun für jeden Knoten Dijkstra aus: \\$\mathcal{O}(|V|(|E| + |V|)\log(|V|))$
    \end{itemize}
    Totale Laufzeit somit: ${\Theta}(|V|(|E| + |V|)\log(|V|))$ mit MinHeap.\\
    Oder ${\Theta(|V|^{2}\log |V|+|V||E|)}$ mit Fibonacci Heap.
\par Super video: \href{https://youtu.be/xc2ua8sQAoE}{https://youtu.be/xc2ua8sQAoE}.
    \vspace*{\stretch{1}}
\end{flashcard}


\end{document}